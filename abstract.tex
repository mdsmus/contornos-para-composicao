Neste artigo descrevemos teorias de contornos, o desenvolvimento,
funcionamento e aprendizado proporcionado pelo \goiaba{}---software
para processamento de contornos---e o uso de combinações de contornos
para criação de obras musicais. Contornos são estruturas que, assim
como conjuntos de notas e motivos, ajudam a dar coerência musical a
uma obra. Apesar de serem bastante abordados pelas áreas de Percepção
e Análise Musical \note{são abordados contornos ou estudos de
combinações de contornos?}, estudos de combinações de contornos para criação
musical são escassos. Assim, este trabalho faz parte de uma pesquisa
maior de investigação do uso de combinações de operações de contornos
melódicos para a composição musical.
