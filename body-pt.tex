\section{Introdução}
\label{sec:introducao}

Contorno pode ser definido como o perfil, desenho ou formato de um
objeto. Pode ser bidimensional e associar altura a comprimento,
largura ou tempo. Em música contornos podem ser associados a altura,
densidade, ritmo, complexidade rítmica, homogeneidade orquestral,
amplitude de harmônicos, intensidade, etc. Contornos melódicos estão
relacionados com movimento de altura em função do tempo.

Há diversas definições de contorno melódico na literatura
\cite{piston59:harmony,toch77:shaping,schonberg:fundamentals,adams76:melodic,marvin.ea87:relating,morris87:composition,clifford95:contour,beard03:contour},
cada uma delas relacionada com o objetivo do trabalhos de seus
autores. No presente estudo preferimos entender que contorno melódico
é um conjunto ordenado de alturas de notas (chamadas de pontos) cujo
valor absoluto é ignorado e somente o registro, que pode ser mantido
ou variar ascendente e descendentemente entre um ponto e outro, é
considerado.

O estudo de contornos é importante porque, assim como conjuntos de
notas e motivos, ajudam a dar coerência musical a uma obra. Eles
representam estruturas musicais manipuláveis através de várias
operações como inversão e retrogradação, e podem ser abordados tanto
do ponto de vista da análise quanto da composição.

No campo da percepção musical contorno melódico é uma importante
característica para o reconhecimento de melodias familiares \cite[p.
136]{dowling.ea86:music}. Além disso é reconhecido que ouvintes têm
maior acuidade na percepção de semelhança de contornos do que na
semelhança de alturas. Por isso novas teorias para comparação de
contornos se tornaram necessárias à área da Análise Musical
\cite[p. 226]{marvin.ea87:relating}.

O presente trabalho faz parte de uma pesquisa maior na qual é proposta
a criação de material composicional a partir de operações com
contornos melódicos. Um software para processar contornos melódicos e
retornar operações pode contribuir significativamente com a citada
pesquisa. Neste artigo pretendemos descrever as funcionalidades e o
estado do desenvolvimento do \goiaba, o software que estamos
desenvolvendo para processar contornos melódicos.

\section{Teorias de contornos}
\label{sec:teorias-de-contornos}

Teorias que visam sistematizar o estudo de contornos dispõem de várias
operações para mapeamento e comparação de contornos
\cite{friedmann85:methodology,friedmann87:response,morris87:composition,morris93:directions,marvin.ea87:relating,clifford95:contour,polansky.ea92:possible,quinn97:fuzzy,beard03:contour}. A
partir das idéias destas teorias é possível, por exemplo, reconhecer
semelhanças entre as duas melodias de quatro notas da figura
\ref{fig:ly-cseg-5968}, cujo contorno está representado em gráfico
cartesiano na figura \ref{fig:cseg-5968}. Outros exemplos de melodias
de contornos semelhantes podem ser vistos nas figuras
\ref{fig:melodias-cseg} e \ref{fig:graficos-cseg}. A terminologia
utilizada nessas figuras é explicada mais adiante, no entanto uma
descrição de todas as operações de contorno foge ao escopo deste
artigo. Assim, descreveremos apenas as operações implementadas no
\goiaba.

\begin{figure}
  \centering
  \subfloat[melodias com cseg $P\langle5\:9\:6\:8\rangle$]{
    \includegraphics[scale=.9]{ly-5968}
    \label{fig:ly-cseg-5968}
  }
  \subfloat[melodias com cseg $Q\langle5\:7\:6\:8\rangle$]{
    \includegraphics[scale=.9]{ly-5768}
    \label{fig:ly-cseg-5768}
  }
  \subfloat[melodias com cseg $R\langle3\:0\:5\:1\rangle$]{
    \includegraphics[scale=.9]{ly-3051}
    \label{fig:ly-cseg-3051}
  }
  \caption{Melodias para diferentes cseg}
  \label{fig:melodias-cseg}
\end{figure}

\begin{figure}
  \centering
  \subfloat[cseg $P\langle5\:9\:6\:8\rangle$]{
    \includegraphics{c-5968}
    \label{fig:cseg-5968}
  }
  \subfloat[cseg $Q\langle5\:7\:6\:8\rangle$]{
    \includegraphics{c-5768}
    \label{fig:cseg-5768}
  }
  \subfloat[cseg $R\langle3\:0\:5\:1\rangle$]{
    \includegraphics{c-3051}

    \label{fig:cseg-3051}
  }
  \caption{Gráficos de cseg}
  \label{fig:graficos-cseg}
\end{figure}

A música pode ser entendida a partir de diferentes espaços, como de
altura e de contorno \cite{morris87:composition}. O espaço de contorno
(\termo{c-space}) é uma abstração de espaço musical que consiste em
elementos organizados do grave para o agudo desconsiderando os
intervalos exatos entre eles. Leva-se em conta apenas a relação entre
os registros dos elementos. O \termo{c-space} pode ser entendido como
um grande conjunto de \termo{c-pitch} (alturas de contorno).  

Cada conjunto de \termo{c-pitch} contido em um \termo{c-space} é
chamado de \termo{cseg} (segmento de contorno)\footnote{Trata-se de
  uma idéia semelhante à da geometria, de reta e segmento de reta,
  onde o \termo{c-space} seria análogo à reta, e o \termo{cseg} ao
  segmento de reta.}. Estes \termo{cseg} podem conter elementos
contíguos ou não do \termo{c-space}. Subconjuntos de \termo{cseg} são
chamados \termo{csubseg}. A figura \ref{fig:c-space} contém três
exemplos de um mesmo \termo{c-space} de 10 \termo{c-pitch} enumerados
do mais grave (0) para o mais agudo (9). Na figura
\ref{fig:c-space5968} o \termo{c-space} contém o \termo{cseg} P, com
os \termo{c-pitch} 5, 9, 6 e 8. Na figura \ref{fig:c-space7420} o
\termo{c-space} contém o \termo{cseg} M, com \termo{c-pitch} 7, 4, 2 e
0. Ainda nesta figura o \termo{cseg} M contém o \termo{csubseg} O, com
\termo{c-pitch} 4 e 2. Por último, na figura \ref{fig:c-space564} o
\termo{c-space} contém o \termo{cseg} N, com os \termo{c-pitch} não
adjacentes 5, 6 e 4.

\begin{figure}
  \centering
  \subfloat[Cseg P]{
    \includegraphics[scale=.97]{cspace-5968}
    \label{fig:c-space5968}
  }
  \subfloat[Cseg M e csubseg O]{
    \includegraphics[scale=.97]{cspace-7420}
    \label{fig:c-space7420}
  }
  \subfloat[Cseg N]{
    \includegraphics[scale=.97]{cspace-564}
    \label{fig:c-space564}
  }
  \caption{C-space com cseg e csubseg}
  \label{fig:c-space}
\end{figure}

Por definição contornos são ordenados no tempo, representados por
letras maiúsculas e têm seus elementos representados por numerais
subscritos. Estes numerais indicam a posição destes elementos em ordem
temporal \cite{marvin.ea87:relating}. Por exemplo, um contorno $P$ de
quatro elementos tem a seguinte configuração: $P\:\langle
P_0\:P_1\:P_2\:P_3\rangle$. Dessa forma, dado um contorno
$P\:\langle5\:9\:6\:8\rangle$, $P_0$ é igual a 5, $P_1$ é igual a 9,
$P_2$ é igual a 6, e $P_3$ é igual a 8.

Operações como retrogradação e rotação são semelhantes às realizadas
com alturas de notas. A expansão de intervalos consiste na
multiplicação de cada \termo{c-pitch} por um dado fator.

A operação de inversão de um contorno $P$ de ordem $q$ é representada
por $IP$, e é matematicamente calculada por
$IP_n=(q-1-P_n)$. Portanto, dado um \termo{cseg}
$P\:\langle5\:9\:6\:8\rangle$ (de ordem 10), $IP_0=(10-1-P_0)$. Logo,
$IP_0=4$. Aplicando-se a mesma idéia aos outros elementos chegamos ao
contorno $IP\:\langle4\:0\:3\:1\rangle$. É possível ainda relacionar a
inversão entre elementos através da matriz de comparação. Calcula-se
esta inversão com a operação $COM(P_a,P_b)=-COM(IP_a,IP_b)$.

Relações de similaridade \cite{marvin.ea87:relating} são analisadas
com operações como comparação (\termo{COM}), matriz de comparação
(\termo{COM-matrix}) e classe de contorno (\termo{CC}).

A operação de comparação \termo{COM(a,b)} retorna a diferença de
registro entre dois elementos, ou seja, informa se um elemento é mais
agudo, mais grave ou de mesma altura que outro. O valor de \termo{COM}
é o sinal ``$+$'' se $b$ é maior que $a$; ``$-$'' se $b$ é maior que
$a$; e ``$0$'' se $b$ é igual a $a$. Por exemplo, no contorno
$P\:\langle5\:9\:6\:8\rangle$, o valor de $COM(P_0,P_1)$ é o sinal
``$+$'', o de $COM(P_1,P_2)$ é ``$-$'', e o valor de $COM(P_3,P_0)$ é
o sinal ``$+$''. Esta medida de comparação pode ser invertida de modo
que a comparação entre dois elementos é igual ao inverso da comparação
destes elementos em ordem reversa. Esta idéia pode ser melhor
entendida observando-se a equação $COM(a,b)=-COM(b,a)$.

A \termo{COM-matrix} é uma matriz bidimensional que compara um
\termo{cseg} com ele próprio. Esta matriz mostra os resultados da
função de comparação \termo{COM} para todos os \termo{c-pitch} de um
\termo{cseg}. A tabela \ref{tab:matriz-5968} contém a matriz de
comparação de um \termo{cseg} $P\:\langle5\:9\:6\:8\rangle$. As
\termo{COM-matrix} dos exemplos já vistos na figura
\ref{fig:graficos-cseg} podem ser vistos na tabela
\ref{tab:matriz-exemplos}.

\begin{table}
  \centering
  \subfloat[cseg $P\langle5\:9\:6\:8\rangle$]{
    \begin{tabular}{c|cccc}
      & $5$ & $9$ & $6$ & $8$ \\
      \hline
      $5$ & $0$ & $+$ & $+$ & $+$ \\
      $9$ & $-$ & $0$ & $-$ & $-$ \\
      $6$ & $-$ & $+$ & $0$ & $+$ \\
      $8$ & $-$ & $+$ & $-$ & $0$
    \end{tabular}
    \label{tab:matriz-5968}
  }
  \qquad
  \subfloat[cseg $Q\langle5\:7\:6\:8\rangle$]{
    \begin{tabular}{c|cccc}
      & $5$ & $7$ & $6$ & $8$ \\
      \hline
      $5$ & $0$ & $+$ & $+$ & $+$ \\
      $7$ & $-$ & $0$ & $-$ & $+$ \\
      $6$ & $-$ & $+$ & $0$ & $+$ \\
      $8$ & $-$ & $-$ & $-$ & $0$
    \end{tabular}
    \label{tab:matriz-5768}
  }
  \qquad
  \subfloat[cseg $R\langle3\:0\:5\:1\rangle$]{
    \begin{tabular}{c|cccc}
      & $3$ & $0$ & $5$ & $1$ \\
      \hline
      $3$ & $0$ & $-$ & $+$ & $-$ \\
      $0$ & $+$ & $0$ & $+$ & $+$ \\
      $5$ & $-$ & $-$ & $0$ & $-$ \\
      $1$ & $+$ & $-$ & $+$ & $0$
    \end{tabular}
    \label{tab:matriz-3051}
  }
  \caption{Exemplos de \termo{COM-matrix}}
  \label{tab:matriz-exemplos}
\end{table}

As diagonais superiores à diagonal principal zero da
\termo{COM-matrix} são chamadas de $INT_n$, onde $n$ é o número da
diagonal: 1 para a superior mais próxima da diagonal zero, 2 para a
seguinte, 3 para a posterior e assim por diante. A diagonal $INT_1$
traz comparações de registros entre elementos adjacentes do
\termo{cseg}. Em $P\:\langle5\:9\:6\:8\rangle$, por exemplo,
$INT_1=\langle+\:-\:+\rangle$, ou seja, o movimento melódico é
ascendente entre 5 e 9, descendente entre 9 e 6, e ascendente entre 6
e 8. Esta comparação é feita também com uma operação conhecida como
série de contornos adjacentes (\termo{CAS}) \footnote{Em teorias de
  contornos não há consenso em relação a terminologia. Por isso há
  idéias semelhantes com nomes diferentes, como $INT_1$ e \termo{CAS}
  \cite{friedmann87:response}.}.

A classe de contorno (\termo{CC}) é uma operação importante para a
verificação de similaridade entre contornos. Obtém-se \termo{CC}
numerando-se ordenadamente todos os \termo{c-pitch} de $0$ a $(n-1)$,
sendo $n$ o número total de \termo{c-pitch} do \termo{cseg}. Uma
\termo{CC} engloba todos os contornos considerados equivalentes. Dois
ou mais contornos são considerados equivalentes quando geram uma mesma
\termo{COM-matrix}, ou seja, quando mantêm a mesma estrutura de
registro entre suas notas. Dessa forma, dado um \termo{cseg}
$P\:\langle5\:9\:6\:8\rangle$, são considerados seus equivalentes os
\termo{cseg} $\langle1\:5\:2\:3\rangle$, $\langle0\:10\:4\:7\rangle$,
$\langle0\:3\:1\:2\rangle$ e muitos outros. Todos eles têm
$\langle0\:3\:1\:2\rangle$ como forma normal e \termo{CC}.

Os Intervalos de Contorno (\termo{CI}) representam as relações entre
\termo{c-pitch} de uma \termo{CC} e podem ser entendidos de duas
formas: guardando o valor entre os \termo{c-pitch}
\cite{friedmann85:methodology}, ou guardando apenas a direção, mas não
o valor da diferença entre os \termo{c-pitch}
\cite{morris93:directions}. Por exemplo, para um mesmo \termo{cseg}
$P\:\langle0\:3\:1\:2\rangle$, o \termo{CI} entre $P_1$ e $P_2$ para
Friedmann é -2, e para Morris, ``$-$''.

O Vetor de Intervalos de Contornos (\termo{CIA}) descreve a freqüência
de cada tipo de \termo{CI} em uma \termo{CC}. Por exemplo, o
\termo{cseg} $P\:\langle0\:3\:1\:2\rangle$ tem \termo{CIA}
$\langle2,1,1/1,1,0\rangle$. Os dígitos da esquerda da barra
representam os \termo{CI} ascendentes em ordem crescente, e os dígitos
da direita os \termo{CI} descendentes em ordem crescente de valor
absoluto. \cite{friedmann85:methodology}. Os Vetores de classe de
contorno (\termo{CCVI} e \termo{CCVII}), de dois dígitos cada um,
refletem os movimentos ascendentes e ascendentes de um contorno e são
calculados a partir dos \termo{CI} ascendentes e descendentes.

A redução de contornos é possível a partir de dois diferentes
algoritmos \cite{adams76:melodic,morris93:directions}. O algoritmo de
Adams, criado para definir uma tipologia de contornos, reduz uma
melodia inteira a apenas quatro \termo{c-pitch}\footnote{Adams utiliza
  uma terminologia diferenciada da que apresentamos aqui.}: a primeira
altura, a última, a mais aguda e a mais grave. O algoritmo de Morris,
processado em várias etapas, reduz a melodia através da observação de
mudanças de direção entre segmentos adjacentes e eliminação de
\termo{c-pitch}.

\section{O Goiaba}
\label{sec:o-software}

O \goiaba{} é desenvolvido em Common Lisp com o compilador SBCL
\cite{team07:sbcl}. A classe \texttt{ponto} define pontos cartesianos
como (x, y), a classe \texttt{contorno-duracao} é uma lista de pontos
como ((x, y) (z, w)), e a classe \texttt{contorno-simples} define
apenas as alturas dos contornos como (y w). Algumas macros de leitura
foram definidas para simplificar o processo de criar instâncias de
objetos. Por exemplo, para criar uma instância do tipo \texttt{ponto}
pode-se fazer normalmente \texttt{(make-instance 'ponto :x 0 :y 3)} ou
simplesmente \verb!#p(0 3)! com a macro de leitura. Da mesma maneira,
um \texttt{contorno-duracao} com dois pontos pode ser instanciado com
\verb!#d(#p(0 3) #p(1 4))! ao invés de usar \texttt{make-instance}.

As operações em contornos são implementadas em métodos, aproveitando
do despacho múltiplo usado pelo sistema de objetos de Common Lisp
(CLOS). Desse modo um método como \texttt{transpor} vai se comportar
de maneira diferente dependendo qual o tipo do seu primeiro parâmetro:

\begin{verbatim}
(defmethod transpor ((objeto contorno-duracao) fator)
  (map-contorno-duracao #L(transpor !1 fator) (pontos objeto)))
\end{verbatim}

\begin{verbatim}
(defmethod transpor ((objeto contorno-simples) fator)
  (map-contorno-simples #L(+ !1 fator) (pontos objeto)))
\end{verbatim}

O \goiaba{} processa contornos representados de forma simples, como
uma lista de \termo{c-pitch}: \verb!#c(5 9 6)!, e contorno com
duração: \verb!#d(#p(0 5)(1 9)(2 6))!. O segundo formato servirá para
a futura implementação do tempo no contorno.

O \goiaba{} tem diversas operações para lidar com contornos, como
tranposição, inversão, retrogradação, rotação e expansão de
intervalos. Além disso operações definidas na literatura também estão
implementadas, como redução de contornos \cite{adams76:melodic},
\termo{CC}, \termo{CAS}, \termo{CASV}, \termo{CI}, \termo{CIA},
\termo{CCVI} e \termo{CCVII} \cite{friedmann85:methodology} e
\termo{COM-matrix} \cite{morris93:directions}.

Finalmente, \goiaba{} usa a biblioteca
Cl-pdf\footnote{www.cliki.net/CL-PDF} para plotar facilmente contornos
definidos, permitindo a fácil visualização de operações em
contornos. As figuras \ref{fig:graficos-cseg} e
\ref{fig:combinacao-operacoes} por exemplo foram criadas
automaticamente pelo \goiaba{}.

Para o futuro pretendemos implementar a possibilidade de gerar
contornos a partir de partituras musicais e vice-versa, e implementar
uma interface gráfica. O desenvolvimento do \goiaba{} segue
procedimentos oriundos do desenvolvimento de software livre como uso
de controle de versão e disponibilização do código-fonte na
internet\footnote{Referência ao código-fonte não inserida por motivo
  do anonimato. Será incluído para a publicação.}.

As descrições das teorias de contornos nem sempre são feitas pelos
seus autores de uma forma totalmente clara. A programação de uma
teoria em um software de computador é um exercício poderoso no
processo de aprendizagem, pois o programador é obrigado a expressar de
forma precisa a compreensão que tem de uma idéia obscura. Dessa forma
a implementação das operações de contorno em um software de computador
ajuda na compreensão da teoria, uma vez que é preciso entender
totalmente como cada operação se processa e que funcionalidade tem.

\section{Uso de contornos na criação musical}
\label{sec:uso-de-contornos}

Contornos são estruturas que, assim como motivos ou conjuntos de
notas, podem ajudar com a coerência musical de uma obra. Para atingir
uma efetividade artística, a coerência e continuidade
musical---princípios que podem operar em vários níveis---não podem
estar limitadas à repetição de elementos, mas devem existir em um
contexto em que tais elementos sejam de alguma maneira diferenciados
\cite[p. 296]{kliewer75:aspects}. Diversas maneiras de diferenciação
de elementos são possíveis através de combinações de operações de
contornos.

Contornos podem ser utilizados em vários níveis de abstração musical,
como nota e duração, motivos, temas, texturas, metas composicionais, e
até no nível da obra inteira. Podem ainda estar associados a vários
parâmetros musicais além de altura, como densidade, dinâmica, timbre e
âmbito. Dessa forma, contornos podem ajudar não apenas na criação e
manipulação de material composicional, mas no planejamento da obra
como um todo, definindo densidade de acordes, perfis de dinâmica, e
estabelecendo pontos estruturais em grande escala. Estudos mostram
como relações de contornos são importantes para a estrutura de várias
obras musicais
\cite{friedmann85:methodology,clifford95:contour,beard03:contour}.

O uso mais interessante de contornos em composição ocorre a partir da
combinação de operações. Pode-se concatenar operações simples como
retrogradação, inversão ou rotação, e operações mais complexas,
definidas nas teorias de contorno. Dessa forma um contorno pode ter
diferentes níveis de diferenciação do seu original. É possível
combinar um grande número de operações. Por exemplo, a figura
\ref{fig:combinacao-operacoes} traz a representação gráfica de um
contorno original $Y\langle1\:3\:2\:0\:4\:5\rangle$ e o resultado de
uma concatenação de operações de inversão, retrogradação, rotação de
fator 3, retrogradação, expansão de intervalos de fator 2 e
transposição de fator 3. O contorno resultante é representado por
$Z\langle9\:7\:11\:3\:5\:13\rangle$.

\begin{figure}
  \centering
  \subfloat[Contorno original $Y\langle1\:3\:2\:0\:4\:5\rangle$]{
    \includegraphics[scale=.9]{c-132045}
  }
  \subfloat[Contorno $Z\langle9\:7\:11\:3\:5\:13\rangle$]{
    \includegraphics[scale=.9]{c-971-11-35-13}
  }
  \caption{Combinação de operações}
  \label{fig:combinacao-operacoes}
\end{figure}

A figura \ref{fig:melodia-concatenacao-operacoes} contém um exemplo
musical composto a partir da combinação de operações. Uma melodia com
uma série de contornos adjacentes \termo{CAS}
$\langle+\:+\:-\:+\rangle$ tem seus intervalos gradativamente
expandidos. Os últimos dois fragmentos têm uma sobreposição de
elementos e ainda uma operação de inversão.

\begin{figure}
  \centering
  \includegraphics[scale=2.8]{melodia-friedmann}
  \caption{Melodia criada a partir da concatenação de operações}
  \label{fig:melodia-concatenacao-operacoes}
\end{figure}

Outro aspecto interessante do uso de contornos melódicos em composição
é o fato de poder ser utilizado sobre qualquer sistema de organização
de alturas, como o sistema modal, tonal, sobre séries e conjuntos de
notas, ou poder ainda estar associado a indeterminação de altura.

\section{Conclusão}
\label{sec:conclusao}

Contornos têm se mostrado úteis para a composição. Durante a pesquisa
percebemos a utilidade de várias operações de contornos para a criação
musical. No entanto ainda é preciso experimentar outras operações. Até
o momento verificamos a utilidade das operações de inversão,
transposição, retrogradação, rotação, expansão de intervalos,
\termo{INT}, redução e preenchimento.

O \goiaba{} tem se mostrado útil tanto para entender contornos quanto
como ferramenta auxiliar à composição musical. Estamos dando
prosseguimento à pesquisa com a revisão das teorias de contornos, a
implementação das operações, a refatoração do \goiaba{}, e a
composição de pequenas peças musicais como experimentos. Esperamos
definir um subconjunto de operações para criar uma interface gráfica e
via linha de comando do \goiaba{}. Assim poderemos lançar a sua
primeira versão e compor uma obra de maior proporção baseada em
combinações de operações de contornos.

%%% Local Variables: 
%%% mode: latex
%%% TeX-master: "goiaba-simples"
%%% End: 
