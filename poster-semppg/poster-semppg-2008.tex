\documentclass{sciposter}
\usepackage[brazil]{babel}
\RequirePackage[utf8,utf8x]{inputenc}
\usepackage{url}
\usepackage{multicol}
\usepackage{graphicx}
\usepackage{color}
\usepackage[dvips,dvipdfm,top=0cm,bottom=0cm,left=0cm,right=6cm,paperwidth=80cm,paperheight=100cm]{geometry}

% usar para termos estrangeiros
\newcommand{\eng}[1]{\textit{#1}}
\newcommand{\goiaba}[1]{\textit{Goiaba}}

\renewcommand{\fontpointsize}{25pt}

\definecolor{SectionCol}{rgb}{1,1,1}
\definecolor{BoxCol}{rgb}{.230,.230,.76}

\def\abovestrut#1{\rule[0in]{0in}{#1}\ignorespaces}
\def\belowstrut#1{\rule[-#1]{0in}{#1}\ignorespaces}

\def\abovespace{\abovestrut{0.20in}}
\def\aroundspace{\abovestrut{0.20in}\belowstrut{0.10in}}
\def\belowspace{\belowstrut{0.10in}}


\title{\goiaba{}: Um software para processar contornos}
\author{\textbf{Marcos Sampaio e Pedro Kröger}}
\institute{Genos---Grupo de pesquisa em computação musical}
\email{mdsmus@gmail.com}

%% inserir nome do orientador
%% inserir área, sub-área e sub-sub-área

% The following commands can be used to alter the default logo settings
\leftlogo[.7]{ufba-logo}{  % defines logo to left of title (with scale factor)
\rightlogo[1.1]{capes-logo}  % same but on right

\begin{document}

\bibliographystyle{plain}

\conference{\Large \textbf{XXVII Seminário Estudantil de Pesquisa---2008
    \hfill \textsf{Grupo de Pesquisa: GENOS}}}

\maketitle

\begin{multicols}{3}

\section{Introdução}

Contorno pode ser definido como o perfil, desenho ou formato de um
objeto. Pode ser bidimensional e associar altura a comprimento,
largura ou tempo. Em música contornos podem ser associados a altura,
densidade, ritmo, complexidade rítmica, homogeneidade orquestral,
amplitude de harmônicos, intensidade, etc. 

Contornos ajudam a dar coerência musical a uma obra assim como
conjuntos de notas e motivos, representam estruturas musicais
manipuláveis através de várias operações como inversão e
retrogradação, e podem ser abordados tanto do ponto de vista da
análise quanto da composição. Dessa forma um software para processar
contornos pode contribuir com a área de Composição Musical. A figura
\ref{fig:cseg-0312} mostra a representação do contorno das duas
melodias da figura \ref{fig:ly-0312}.

\section{Teorias de contorno}

Existem várias operações de mapeamento e comparação de contornos nas
teorias desenvolvidas para sistematização do estudo de contornos.
\cite{friedmann85:methodology,friedmann87:response,morris87:composition,morris93:directions,marvin.ea87:relating,clifford95:contour,polansky.ea92:possible,quinn97:fuzzy,beard03:contour}.

Utilizamos o conceito que considera contorno \textbf{um conjunto
  ordenado de elementos distintos, com ou sem repetição, numerados de
  forma ascendente} \cite[p. 206]{morris93:directions}. Esta numeração
é feita atribuindo 0 para o elemento de menor valor, 1 para o segundo
elemento de menor valor, 2 para o terceiro elemento de menor valor, e
assim por diante. O valor do elemento depende do parâmetro que ele
mapeia. Por exemplo, em um contorno de alturas a nota mais grave tem o
menor valor. Em um contorno de densidade de acordes, o acorde de menor
número de notas tem o menor valor.

A partir dos conceitos e operações destas teorias é possível, por
exemplo, reconhecer semelhanças entre as alturas das duas melodias m1
e m2 da figura \ref{fig:ly-0312}, cujo contorno está representado
graficamente na figura \ref{fig:cseg-0312} e simbolicamente por (0 3 1
2).  As semelhanças entre estas melodias da figura \ref{fig:ly-0312}
são identificadas apenas com a observação dos seus contornos.

As operações de teorias de contorno são inversão, retrogradação,
comparação e matriz de comparação de valores de elementos
\cite{morris87:composition}, comparação entre elementos adjacentes e
não adjacentes, classe de contornos equivalentes
\cite{marvin.ea87:relating}, forma prima, forma normal, redução
\cite{adams76:melodic},

\section{O Goiaba}

\end{multicols}

\begin{center}

\begin{multicols}{2}

  \begin{figure}
    \centering
    \includegraphics[scale=2]{ly-0312}
    \caption{Melodias de contorno P(0 3 1 2)}
    \label{fig:ly-0312}
  \end{figure}

  \begin{figure}
    \centering
    \includegraphics[scale=2]{c-0312}
    \caption{Representações gráficas do contorno P(0 3 1 2)}
    \label{fig:cseg-0312}
  \end{figure}
  
\end{multicols}

\end{center}

\begin{multicols}{3}

\section{Contornos em Composição}

\section{Conclusão}

\renewcommand{\refname}{Bibliografia}

\bibliography{melodic-contour,music-harmony-and-theory,composition}

\end{multicols}

\end{document}