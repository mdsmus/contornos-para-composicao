\documentclass{sciposter}
\usepackage[brazil]{babel}
\RequirePackage[utf8,utf8x]{inputenc}
\usepackage{url}
\usepackage{multicol}
\usepackage{graphicx}
\usepackage{color}

% usar para termos estrangeiros
\newcommand{\eng}[1]{\textit{#1}}
\newcommand{\goiaba}[1]{\textit{Goiaba}}

\renewcommand{\papertype}{custom}
\renewcommand{\fontpointsize}{25pt}

\setlength{\paperwidth}{80cm}
\setlength{\paperheight}{100cm}
\renewcommand{\setpspagesize}{
  \ifthenelse{\equal{\orientation}{portrait}}{
    \special{papersize=80cm,100cm}
    }{\special{papersize=100cm,80cm}
    }
  }

\definecolor{SectionCol}{rgb}{1,1,1}
\definecolor{BoxCol}{rgb}{.230,.230,.76}

\def\abovestrut#1{\rule[0in]{0in}{#1}\ignorespaces}
\def\belowstrut#1{\rule[-#1]{0in}{#1}\ignorespaces}

\def\abovespace{\abovestrut{0.20in}}
\def\aroundspace{\abovestrut{0.20in}\belowstrut{0.10in}}
\def\belowspace{\belowstrut{0.10in}}


\title{\goiaba{}: Um software para processar contornos}
\author{\textbf{Marcos Sampaio e Pedro Kröger}}
\institute{Genos---Grupo de pesquisa em computação musical}
\email{mdsmus@gmail.com}

%% inserir nome do orientador
%% inserir área, sub-área e sub-sub-área

% The following commands can be used to alter the default logo settings
\leftlogo[.7]{ufba-logo}{  % defines logo to left of title (with scale factor)
\rightlogo[1.1]{capes-logo}  % same but on right

\begin{document}

\bibliographystyle{plain}

\conference{\Large \textbf{XXVII Seminário Estudantil de Pesquisa---2008
    \hfill \textsf{Grupo de Pesquisa: GENOS}}}

\maketitle

\begin{multicols}{3}

\section{Introdução}

Contorno pode ser definido como o perfil, desenho ou formato de um
objeto. Pode ser bidimensional e associar altura a comprimento,
largura ou tempo. Em música contornos podem ser associados a altura,
densidade, ritmo, complexidade rítmica, homogeneidade orquestral,
amplitude de harmônicos, intensidade, etc. Contornos melódicos
associam altura a tempo.

No presente estudo entendemos que contorno melódico é um conjunto
ordenado de alturas de notas (chamadas de pontos) cujo valor absoluto
é ignorado e somente o registro, que pode ser mantido ou variar
ascendente e descendentemente entre um ponto e outro, é considerado.

O estudo de contornos é importante porque, assim como conjuntos de
notas e motivos, contornos ajudam a dar coerência musical a uma
obra. Eles representam estruturas musicais manipuláveis através de
várias operações como inversão e retrogradação, e podem ser abordados
tanto do ponto de vista da análise quanto da composição. Dessa forma
um software para processar contornos pode contribuir com a área de
Composição Musical.

\section{Teorias de contorno}

Teorias que visam sistematizar o estudo de contornos dispõem de várias
operações para mapeamento e comparação de contornos
\cite{friedmann85:methodology,friedmann87:response,morris87:composition,morris93:directions,marvin.ea87:relating,clifford95:contour,polansky.ea92:possible,quinn97:fuzzy,beard03:contour}
. A partir das idéias destas teorias é possível, por exemplo,
reconhecer semelhanças entre as duas melodias de quatro notas da
figura \ref{fig:ly-cseg-5968}, cujo contorno está representado em
gráfico cartesiano na figura \ref{fig:cseg-5968}.

\section{O Goiaba}

\end{multicols}

\begin{center}

\begin{multicols}{2}

\begin{figure}
  \centering
  \includegraphics[scale=2]{ly-5968}
  \label{fig:ly-cseg-5968}
  \caption{Melodias com contorno (5 9 6 8)}
\end{figure}
\begin{figure}
  \centering
  \includegraphics[scale=2]{ly-5768}
  \label{fig:ly-cseg-5768}
  \caption{melodias com contorno (5 7 6 8)}
}
\end{figure}
\begin{figure}
  \centering
    \includegraphics[scale=2]{ly-3051}
    \label{fig:ly-cseg-3051}
    \caption{melodias com contorno (3 0 5 1)}
  \label{fig:melodias-cseg}
\end{figure}

\begin{figure}
  \centering
  \includegraphics[scale=2]{c-5968}
  \label{fig:ly-cseg-5968}
  \caption{gráficos com contorno (5 9 6 8)}
\end{figure}
\begin{figure}
  \centering
  \includegraphics[scale=2]{c-5768}
  \label{fig:ly-cseg-5768}
  \caption{gráficos com contorno (5 7 6 8)}
}
\end{figure}
\begin{figure}
  \centering
    \includegraphics[scale=2]{c-3051}
    \label{fig:ly-cseg-3051}
    \caption{gráficos com contorno (3 0 5 1)}
  \label{fig:melodias-cseg}
\end{figure}

\end{multicols}

\end{center}

\begin{multicols}{3}

\section{Contornos em Composição}

\section{Conclusão}

\renewcommand{\refname}{Bibliografia}

\bibliographystyle{plain}
\bibliography{melodic-contour,music-harmony-and-theory,composition}

\end{multicols}

\end{document}