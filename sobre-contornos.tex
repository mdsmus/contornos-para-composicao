\documentclass{article}
\usepackage{anppom2008}

% usar para termos estrangeiros
\newcommand{\eng}[1]{\textit{#1}}

% usar para nomes de obras
\newcommand{\opus}[1]{\textit{#1}}

% usar para nomes de termos
\newcommand{\termo}[1]{\textit{#1}}

\newcommand{\ok}{
  \multicolumn{1}{>{\columncolor[gray]{.6}}c}{}
}

\newcommand{\tri}[1]{
  #1\textsuperscript{o} t
}


\begin{document}
\graphicspath{{figs-out/}{out/}}

\title{Software para processar contornos}
\author{Marcos da Silva Sampaio e Pedro Kröger}{Universidade Federal
  da Bahia}{{mdsmus,pedro.kroeger}@gmail.com}{genos.mus.br}

\begin{sumario}
  Resumo do texto com até 100 palavras.  
\end{sumario}

\keywords{Contornos, Computação musical, Composição, Lisp}

\section{Introdução}
\label{sec:introducao}

\renewcommand{\refname}{Referências Bibliográficas}
\bibliography{music-harmony-and-theory,music-history,melodic-contour,music-analysis,music-scores,music-perception}
\bibliographystyle{kchicago}

\end{document}
