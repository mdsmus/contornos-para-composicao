\documentclass{article}
\usepackage{ifthen}
\usepackage[utf8,utf8x]{inputenc}
\usepackage[a4paper,top=2cm,bottom=2cm,left=2cm,right=2cm]{geometry}
\usepackage[brazil]{babel}
\usepackage{setspace}
\usepackage{graphicx}
\usepackage{url}
\usepackage{colortbl}

% usar para termos estrangeiros
\newcommand{\eng}[1]{\textit{#1}}

% usar para nomes de obras
\newcommand{\opus}[1]{\textit{#1}}

% usar para nomes de termos
\newcommand{\termo}[1]{\textit{#1}}

\newcommand{\ok}{
  \multicolumn{1}{>{\columncolor[gray]{.6}}c}{}
}

\newcommand{\tri}[1]{
  #1\textsuperscript{o} t
}

\title{Sobre Contornos}
\author{Marcos di Silva}

\begin{document}

\setlength{\parindent}{0cm}
\maketitle
\thispagestyle{empty}

Friedmann \cite{friedmann85:_method_discus_contour} afirma que a
música do século XX tem sido em grande parte analisada em função de
relações de classes de alturas. Ele defende o uso independente de
contornos melódicos para a análise deste repertório. Argumenta que
ouvintes têm uma maior acuidade com contornos que com relações de
classes de altura. Exemplifica a eficiência do uso de contornos com
obras de Schönberg que têm no contorno o elemento mais
importante. Propõe ferramentas e operações para a autonomia de uma
análise a partir de contornos melódicos.

De acordo com Marvin e Laprade \cite{marvin87:_relat_music_contour}
teóricos reconhecem que ouvintes estão mais aptos a perceber
semelhanças de contornos que semelhanças entre relações de altura. Por
conta disso novas teorias foram necessárias para comparação de
contornos.

Friedmann \cite{friedmann85:_method_discus_contour} propõe as seguintes
ferramentas:

\paragraph{Contour Adjacency Series (CAS)}
\label{sec:cont-adjac-seri}

Série de elementos adjacentes de um contorno. É uma série de sinais
'+' e '-' referentes às inclinações positivas e negativas entre notas
adjascentes de um contorno. Contornos adjacentes de mesma altura são
desconsiderados, assim como não há uma indicação de repetição de
altura em larga escala. Por exemplo, para um contorno ``d c f d'' CAS
= (-,+,-); para um contorno ``d c c e'', CAS = (- +).

A principal utilidade da CAS está na comparação de equivalência com
retrógrado ou rotação de ordem dos elementos do contorno.

\paragraph{Contour Class (CC)}
\label{sec:contour-class-cc}

Classe de contornos. É uma série ordenada de números que representa a
relação de alturas entre todas as notas de um contorno, na qual a nota
mais grave é indicada por 0 e a nota mais aguda por n-1, onde n=total
de notas. Por exemplo, para um contorno ``d c f e'' CC = (1 0 3 2).

\paragraph{Contour Adjacency Series Vector (CASV)}
\label{sec:cont-adjac-seri-1}

Vetor de classe de contornos adjacentes. É um vetor de dois dígitos
com a soma das inclinações positivas e negativas de um contorno. Por
exemplo, para um contorno ``d c f e'' CASV = (1,2).

Sua principal utilidade é estabelecer uma classe de equivalência entre
rotações, retrógrados e outras permutações de ordem de elementos.

\paragraph{Contour Interval (CI)}
\label{sec:contour-interval-ci}

Intervalo de contorno. É a distância entre dois elementos de uma
classe de contorno (CC). Por exemplo, em CC (1 0 3 2), a CI de 1-0 é
-1, a CI de 0-3 é +3.

\paragraph{Contour Interval Succession (CIS)}
\label{sec:cont-interv-succ}

Sucessão de intervalos de contorno é uma série de números que
representam os intervalos de contorno (CI) de uma classe de contorno
(CC). Por exemplo, em CC (1 0 3 2), CI = (1 3 -1).

\paragraph{Contour Interval Array (CIA)}
\label{sec:cont-interv-array}

%% FIXME: o termo array pode ser traduzido por vetor neste contexto?
Vetor de intervalos de contorno. É uma série de números que representa
a multiplicidade de intervalos de contornos (CI) de uma classe de
contorno (CC). Esta série é dividida em duas e cada número representa
o número de CI's de valor 1, 2, 3, etc, a depender do tamanho do
contorno. A primeira parte da série representa CI's positivos e a
segunda representa CI's negativos. Por exemplo, para CC = (1 0 3 2),
CIA = (1 2 1 , 2 0 0). Esta CC tem um intervalo de valor +1, dois de
valor +2, um de valor +3, dois de valor -1 e nenhum de valores -2 e
-3.

\paragraph{Contour Class Vector I (CCVI)}
\label{sec:contour-class-vector-1}

Vetor de classe de contorno I. É um par de números que representa o
grau de ascendência e descendência de uma classe de contorno (CC). É
obtido através da soma da multiplicação do número de ocorrências de
cada intervalo de contorno (CI) pelo próprio valor do intervalo. Por
exemplo, em um CIA ((1 2 1)(2 0 0)) CCVI = (8 2).

\paragraph{Contour Class Vector II (CCVII)}
\label{sec:contour-class-vector-2}

Vetor de classe de contorno II. É um par de números que representa o
grau de ascendência e descendência de uma classe de contorno (CC). É
obtido através da soma dos valores positivos e negativos das
ocorrências de intervalos de contorno (CI). Por exemplo, em um CIA ((1
2 1)(2 0 0)), CCVII = (8 2)

Autores diferentes deram nomes diferentes para ferramentas de função
semelhante. Friedmann \cite{friedmann1987rmc} discute a respeito das
ferramentas criadas por Marvin e Laprade
\cite{marvin87:_relat_music_contour}, Morris \cite{morris1987cpc} e
ele próprio \cite{friedmann85:_method_discus_contour}.

\bibliography{mestrado}
\bibliographystyle{plainurl-br}

\end{document}
