\documentclass{article}
\usepackage{anppom2008}

% usar para termos estrangeiros
\newcommand{\eng}[1]{\textit{#1}}

% usar para nomes de obras
\newcommand{\opus}[1]{\textit{#1}}

% usar para nomes de termos
\newcommand{\termo}[1]{\textit{#1}}

\newcommand{\ok}{
  \multicolumn{1}{>{\columncolor[gray]{.6}}c}{}
}

\newcommand{\tri}[1]{
  #1\textsuperscript{o} t
}

\newcommand{\goiaba}{\opus{Goiaba}}

\begin{document}
\graphicspath{{figs-out/}{out/}}

\title{Software para processar contornos}
\author{Marcos da Silva Sampaio e Pedro Kröger}{Universidade Federal
  da Bahia}{{mdsmus,pedro.kroeger}@gmail.com}{genos.mus.br}

\begin{sumario}
  Resumo do texto com até 100 palavras.  
\end{sumario}

\keywords{Contornos, Computação musical, Composição, Lisp}

%% divisão de trabalho

%% por que contornos são importantes
%%%% na análise
%%%% na composição

%% criação de um programa para ajudar na compreensão de contornos

%% explicar as teorias de contornos

%% dar uma visão geral do programa
%%%% mostrar combinações

%% falar sobre criação de material a partir de operações e falar de
%% operações de contorno

\section{Introdução}
\label{sec:introducao}

Contorno pode ser definido como o perfil, desenho ou formato de um
objeto. Pode ser bidimensional e associar altura a comprimento,
largura ou tempo. Em música contornos podem ser associados a altura,
densidade, ritmo, complexidade rítmica, homogeneidade orquestral,
amplitude de harmônicos, intensidade, etc. Contornos melódicos estão
relacionados com movimento de altura em função do tempo.

Há diversas definições de contorno melódico na literatura
\cite{piston59:harmony,toch77:shaping,schonberg:fundamentals,adams76:melodic,marvin.ea87:relating,morris87:composition,clifford95:contour,beard03:contour}. Cada
uma destas definições está relacionada com o objetivo do trabalhos de
seus autores. No presente estudo preferimos entender que contorno
melódico é o conjunto ordenado dos movimentos ascendentes,
descendentes, e nulos entre as notas de uma melodia.

%% por que contornos são importantes

O estudo de contornos é importante porque, assim como conjuntos de
notas e motivos, contornos ajudam a dar coerência musical a uma
obra. Eles representam estruturas musicais manipuláveis através de
várias operações como inversão e retrogradação. Contornos podem ser
abordados tanto do ponto de vista da análise, quanto da composição.

Na música pré-serial de Webern, por exemplo, na falta de outros
sistemas dominantes de organização de alturas, contorno representa um
fator estrutural igual em importância às relações da Teoria dos
Conjuntos \cite[p. 157]{clifford95:contour}.

A idéia de preservação de contorno e variação de intervalos entre
notas é encontrada em diferentes situações musicais. Há adequação de
notas à tonalidade em respostas tonais de fugas, em mudanças de modo
em peças do tipo tema e variações, em \eng{leitmotif} e idéias fixas,
citando apenas alguns exemplos
\cite[p. 29]{morris87:composition}. Outras obras têm motivos cujos
intervalos são pouco a pouco expandidos ou contraídos, como por
exemplo ocorre no início da \opus{Música para Cordas, Percussão e
  Celesta}, de Béla Bartók.

No campo da percepção musical contorno melódico é uma importante
característica para o reconhecimento de melodias familiares
\cite[p. 136]{dowling.ea86:music}. Alguns teóricos reconhecem que
ouvintes têm maior acuidade na percepção de semelhança de contornos do
que na semelhança de alturas. Por isso novas teorias para comparação
de contornos se tornaram necessárias à área da Análise Musical
\cite[p. 226]{marvin.ea87:relating}.

%% criação de um programa para ajudar na compreensão de contornos

O presente trabalho faz parte de uma pesquisa maior na qual é proposta
a criação de material composicional a partir de operações com
contornos melódicos. Um software para processar contornos melódicos e
retornar operações pode contribuir significativamente com a citada
pesquisa. O Processador de Classe de Notas \cite{oliveira01:pcn} é um
exemplo de software criado para o mesmo fim.

\section{Teorias de contornos}
\label{sec:teorias-de-contornos}

%% FIXME: remover informação histórica
Teorias que visam sistematizar o estudo de contornos dispõem de várias
operações para mapeamento e comparação de contornos
\cite{friedmann85:methodology,friedmann87:response,morris87:composition,morris93:directions,marvin.ea87:relating,clifford95:contour,polansky.ea92:possible,quinn97:fuzzy,beard03:contour}.

A música pode ser entendida a partir de diferentes espaços, como de
altura e de contorno \cite{morris87:composition}. O espaço de contorno
(\eng{c-space}) é uma abstração de espaço musical que consiste em
elementos organizados do grave para o agudo desconsiderando os
intervalos exatos entre eles.

Uma composição pode ser entendida como uma matriz de características
rítmicas, de classe de notas, de contornos e de timbre. As relações
entre contorno e classe de notas podem ser abordadas de duas maneiras:
considerando possíveis associações entre conjuntos e contornos, ou
tratando conjuntos de notas e contornos de forma autônoma
\cite{friedmann85:methodology}.

Contornos são representados por letras maiúsculas e seus elementos por
numerais subscritos. Estes numerais representam a posição destes
elementos em ordem temporal de aparecimento no contorno. Por exemplo,
em um contorno $P\:\langle5\:9\:6\:8\rangle$, $P_0 = 5$, $P_1 = 9$,
$P_2 = 6$, e $P_3 = 8$.

\section{O software}
\label{sec:o-software}

\section{Programando para aprender}
\label{sec:progr-para-aprend}

\section{Composição com contornos}
\label{sec:comp-com-cont}

\renewcommand{\refname}{Referências Bibliográficas}
\bibliography{music-harmony-and-theory,music-history,melodic-contour,music-analysis,music-scores,music-perception,composition,programs}
\bibliographystyle{kchicago}

\end{document}
