\documentclass[brazil]{article}
\usepackage{anppom2008}
\usepackage{color}

\newcommand{\eng}[1]{\textit{#1}}
\newcommand{\opus}[1]{\textit{#1}}
\newcommand{\termo}[1]{\textit{#1}}
\newcommand{\goiaba}{\opus{Goiaba}}
\newcounter{notacounter}
\newcommand{\note}[1]{
  \addtocounter{notacounter}{1}
  \textcolor{red}{[nota \arabic{notacounter}: #1]}
}

\begin{document}
\graphicspath{{figs/}{out/}{data/}{lily/}}

\title{Software para processar contornos}
\author{}{}{}{}
%\author{Marcos da Silva Sampaio e Pedro Kröger}{Universidade Federal
%  da Bahia}{{mdsmus,pedro.kroeger}@gmail.com}{genos.mus.br}

\begin{sumario}
  Resumo do texto com até 100 palavras.  
\end{sumario}

\keywords{Contornos, Computação musical, Composição, Lisp}

%% divisão de trabalho

%% por que contornos são importantes
%%%% na análise
%%%% na composição

%% criação de um programa para ajudar na compreensão de contornos

%% explicar as teorias de contornos

%% dar uma visão geral do programa
%%%% mostrar combinações

%% falar sobre criação de material a partir de operações e falar de
%% operações de contorno

\section{Introdução}
\label{sec:introducao}

Contorno pode ser definido como o perfil, desenho ou formato de um
objeto. Pode ser bidimensional e associar altura a comprimento,
largura ou tempo. Em música contornos podem ser associados a altura,
densidade, ritmo, complexidade rítmica, homogeneidade orquestral,
amplitude de harmônicos, intensidade, etc. Contornos melódicos estão
relacionados com movimento de altura em função do tempo.

Há diversas definições de contorno melódico na literatura
\cite{piston59:harmony,toch77:shaping,schonberg:fundamentals,adams76:melodic,marvin.ea87:relating,morris87:composition,clifford95:contour,beard03:contour},
cada uma delas relacionada com o objetivo do trabalhos de seus
autores. No presente estudo preferimos entender que contorno melódico
é um conjunto ordenado dos movimentos ascendentes, descendentes, e
nulos entre as notas de uma melodia. \note{Melhorar definição de
  contorno: ``conjunto dos movimentos ascendentes, desdendentes e do
  não movimento entre notas de uma melodia''?}

%% por que contornos são importantes

O estudo de contornos é importante porque, assim como conjuntos de
notas e motivos, ajudam a dar coerência musical a uma obra. Eles
representam estruturas musicais manipuláveis através de várias
operações como inversão e retrogradação, e podem ser abordados tanto
do ponto de vista da análise quanto da composição.

A idéia de preservação de contorno e variação de intervalos entre
notas é encontrada em diferentes situações musicais. Há adequação de
notas à tonalidade em respostas tonais de fugas, em mudanças de modo
em peças do tipo tema e variações, em \eng{leitmotif} e idéias fixas,
citando apenas alguns exemplos
\cite[p. 29]{morris87:composition}. Outras obras têm motivos cujos
intervalos são pouco a pouco expandidos ou contraídos, como por
exemplo ocorre no início da \opus{Música para Cordas, Percussão e
  Celesta}, de Béla Bartók. \note{nao esta clara a relacao dessa obra
  com contornos. marcos: ela não está relacionada com contornos, mas
  com expansão de intervalos}

No campo da percepção musical contorno melódico é uma importante
característica para o reconhecimento de melodias familiares \cite[p.
136]{dowling.ea86:music}. É reconhecido que ouvintes têm maior
acuidade na percepção de semelhança de contornos do que na semelhança
de alturas. Por isso novas teorias para comparação de contornos se
tornaram necessárias à área da Análise Musical
\cite[p. 226]{marvin.ea87:relating}.

%% criação de um programa para ajudar na compreensão de contornos

O presente trabalho faz parte de uma pesquisa maior na qual é proposta
a criação de material composicional a partir de operações com
contornos melódicos. Um software para processar contornos melódicos e
retornar operações pode contribuir significativamente com a citada
pesquisa. Com este artigo pretendemos descrever as funcionalidades e o
estado do desenvolvimento do Goiaba, o software que estamos
desenvolvendo para processar contornos melódicos.

\section{Teorias de contornos}
\label{sec:teorias-de-contornos}

Teorias que visam sistematizar o estudo de contornos dispõem de várias
operações para mapeamento e comparação de contornos
\cite{friedmann85:methodology,friedmann87:response,morris87:composition,morris93:directions,marvin.ea87:relating,clifford95:contour,polansky.ea92:possible,quinn97:fuzzy,beard03:contour}. Estas
teorias permitem, por exemplo, considerar semelhantes as melodias da
figura \ref{fig:ly-cseg-5968}, representadas em gráfico cartesiano na
figura \ref{fig:cseg-5968}. Outros exemplos podem ser vistos nas
figuras \ref{fig:graficos-cseg} e \ref{fig:melodias-cseg}. A
terminologia é explicada mais adiante.

\begin{figure}
  \centering
  \subfigure[melodias com cseg $R\langle5\:9\:6\:8\rangle$]{
    \includegraphics{ly-5968}
    \label{fig:ly-cseg-5968}
  }
  \subfigure[melodias com cseg $R\langle5\:7\:6\:8\rangle$]{
    \includegraphics{ly-5768}
    \label{fig:ly-cseg-5768}
  }
  \subfigure[melodias com cseg $R\langle3\:0\:5\:1\rangle$]{
    \includegraphics{ly-3051}
    \label{fig:ly-cseg-3051}
  }
  \caption{Melodias para diferentes cseg}
  \label{fig:melodias-cseg}
\end{figure}

\begin{figure}
  \centering
  \subfigure[cseg $R\langle5\:9\:6\:8\rangle$]{
    \includegraphics{c-5968}
    \label{fig:cseg-5968}
  }
  \subfigure[cseg $R\langle5\:7\:6\:8\rangle$]{
    \includegraphics{c-5768}
    \label{fig:cseg-5768}
  }
  \subfigure[cseg $R\langle3\:0\:5\:1\rangle$]{
    \includegraphics{c-3051}

    \label{fig:cseg-3051}
  }
  \caption{Gráficos de cseg}
  \label{fig:graficos-cseg}
\end{figure}

A música pode ser entendida a partir de diferentes espaços, como de
altura e de contorno \cite{morris87:composition}. O espaço de contorno
(\termo{c-space}) é uma abstração de espaço musical que consiste em
elementos organizados do grave para o agudo desconsiderando os
intervalos exatos entre eles. O \termo{c-space} pode ser entendido
como um grande conjunto de \termo{c-pitch} (alturas de contorno). Cada
contorno contido em um \termo{c-space} é chamado de \termo{cseg}
(segmento de contorno)\footnote{Trata-se de uma idéia semelhante à da
  geometria, de reta e segmento de reta, onde o \termo{c-space} seria
  análogo à reta, e o \termo{cseg} ao segmento de reta.}. Subconjuntos
de \termo{cseg} são chamados \termo{csubseg}, e podem conter elementos
contíguos ou não do \termo{cseg} que os contém. A figura
\ref{fig:c-space} contém três exemplos de um mesmo \termo{c-space} de
10 \termo{c-pitch} enumerados do mais grave (0) para o mais agudo
(9). Na figura \ref{fig:c-space5968} o \termo{c-space} contém o
\termo{cseg} P, com os \termo{c-pitch} 5, 9, 6 e 8. Na figura
\ref{fig:c-space7420} o \termo{c-space} contém o \termo{cseg} M, com
\termo{c-pitch} 7, 4, 2 e 0. Ainda nesta figura o \termo{cseg} M
contém o \termo{csubseg} O, com \termo{c-pitch} 4 e 2. Por último, na
figura \ref{fig:c-space564} o \termo{c-space} contém o \termo{cseg} N,
com os \termo{c-pitch} não adjacentes 5, 6 e 4.

\begin{figure}
  \centering
  \subfigure[Cseg P]{
    \includegraphics{cspace-5968}
    \label{fig:c-space5968}
  }
  \subfigure[Cseg M e csubseg O]{
    \includegraphics{cspace-7420}
    \label{fig:c-space7420}
  }
  \subfigure[Cseg N]{
    \includegraphics{cspace-564}
    \label{fig:c-space564}
  }
  \caption{C-space com cseg e csubseg}
  \label{fig:c-space}
\end{figure}

Por definição contornos são ordenados no tempo, e representados por
letras maiúsculas e seus elementos por numerais subscritos. Estes
numerais representam a posição destes elementos em ordem temporal de
aparecimento no contorno \cite{marvin.ea87:relating}. Por exemplo, em
um contorno $P\:\langle5\:9\:6\:8\rangle$, $P_0$ é igual a 5, $P_1$ é
igual a 9, $P_2$ é igual a 6, e $P_3$ é igual a 8.

Relações de similaridade \cite{marvin.ea87:relating} são analisadas
com a função de comparação \termo{COM}, com a matriz de comparação
(\termo{COM-matrix}), com as formas normal e prima, com a classe de
contorno (\termo{CC}), com a função de similaridade \termo{CSIM}, e
com a função de contorno embutido \termo{CEMB}. As operações com
contorno são identidade (P), inversão (I), retrogradação (R),
retrogradação da inversão (RI) e translação.

%% FIXME: tirar parênteses da referência à equação
A função de comparação \termo{COM} mede a diferença de registro entre
dois elementos, ou seja, informa se um elemento é mais agudo, mais
grave ou de mesma altura que outro. O valor de \termo{COM} é o sinal
``$+$'' se $a$ é menor que $b$; ``$-$'' se $a$ é maior que $b$; e
``$0$'' se $a$ é igual a $b$. Por exemplo, no contorno
$P\:\langle5\:9\:6\:8\rangle$, o valor de $COM(P_0,P_1)$ é o sinal
``$+$'', o de $COM(P_1,P_2)$ é ``$-$'', e o valor de $COM(P_3,P_0)$ é
o sinal ``$+$''. Esta medida de comparação pode ser invertida de modo
que a comparação entre dois elementos é igual ao inverso da comparação
destes elementos em ordem reversa. Esta idéia pode ser melhor
entendida observando-se a equação $COM(a,b)=-COM(b,a)$.

A \termo{COM-matrix} é uma matriz bidimensional que compara um
\termo{cseg} com ele próprio. Ela mostra os resultados da função de
comparação \termo{COM} para todos os \termo{c-pitch} de um
\termo{cseg}. Cada posição da matriz é representada de modo genérico
por $E_(P_x,P_y)$, onde $E$ representa a própria matriz, $P$
representa o \termo{cseg} que dá origem à matriz, $P_x$ representa
genericamente um \termo{c-pitch} do \termo{cseg} $P$ localizado
horizontalmente na matriz, e $P_y$ representa um \termo{c-pitch} de
$P$ localizado verticalmente na matriz. A tabela \ref{tab:matriz-5968}
contém a matriz de comparação de um \termo{cseg}
$P\:\langle5\:9\:6\:8\rangle$, e outros exemplos de \termo{COM-matrix}
podem ser vistos na tabela \ref{tab:matriz-exemplos}.

\begin{table}
  \centering
  \subtable[cseg $P\langle5\:9\:6\:8\rangle$]{
    \begin{tabular}{c|cccc}
      & $5$ & $9$ & $6$ & $8$ \\
      \hline
      $5$ & $0$ & $+$ & $+$ & $+$ \\
      $9$ & $-$ & $0$ & $-$ & $-$ \\
      $6$ & $-$ & $+$ & $0$ & $+$ \\
      $8$ & $-$ & $+$ & $-$ & $0$
    \end{tabular}
    \label{tab:matriz-5968}
  }
  \qquad
  \subtable[cseg $Q\langle5\:7\:6\:8\rangle$]{
    \begin{tabular}{c|cccc}
      & $5$ & $7$ & $6$ & $8$ \\
      \hline
      $5$ & $0$ & $+$ & $+$ & $+$ \\
      $7$ & $-$ & $0$ & $-$ & $+$ \\
      $6$ & $-$ & $+$ & $0$ & $+$ \\
      $8$ & $-$ & $-$ & $-$ & $0$
    \end{tabular}
    \label{tab:matriz-5968}
  }
  \qquad
  \subtable[cseg $R\langle3\:0\:5\:1\rangle$]{
    \begin{tabular}{c|cccc}
      & $3$ & $0$ & $5$ & $1$ \\
      \hline
      $3$ & $0$ & $-$ & $+$ & $-$ \\
      $0$ & $+$ & $0$ & $+$ & $+$ \\
      $5$ & $-$ & $-$ & $0$ & $-$ \\
      $1$ & $+$ & $-$ & $+$ & $0$
    \end{tabular}
    \label{tab:matriz-3051}
  }
  \caption{Exemplos de \termo{COM-matrix}}
  \label{tab:matriz-exemplos}
\end{table}

Esta matriz tem uma diagonal principal zero. Esta diagonal estabelece
uma simetria entre elementos da matriz. As diagonais superiores são
chamadas de $INT_n$, onde $n$ é o número da diagonal: 1 para a
superior mais próxima da diagonal zero, 2 para a seguinte, 3 para a
posterior e assim por diante. A diagonal $INT_1$ traz comparações de
registros entre elementos adjacentes do \termo{cseg}. Em
$P\:\langle5\:9\:6\:8\rangle$, por exemplo,
$INT_1=\langle+\:-\:+\rangle$, ou seja, o movimento melódico é
ascendente entre 5 e 9, descendente entre 9 e 6, e ascendente entre 6
e 8. Esta comparação é feita também com uma operação conhecida como
\termo{CAS} \footnote{Em teorias de contornos não há consenso em
  relação a terminologia. Por isso há idéias semelhantes com nomes
  diferentes, como $INT_1$ e \termo{CAS}
  \cite{friedmann87:response}.}.

A classe de contorno (\termo{CC}) é uma operação importante para a
verificação de similaridade entre contornos. Assim como a forma
normal, a \termo{CC} é obtida numerando-se ordenadamente todos os
\termo{c-pitch} de $0$ a $(n-1)$, sendo $n$ o número de
\termo{c-pitch} do \termo{cseg}. Uma \termo{CC} engloba todos os
contornos considerados equivalentes.

Dois ou mais contornos são considerados equivalentes se geram uma
mesma \termo{COM-matrix}, ou seja, se mantêm a mesma estrutura de
registro entre suas notas. Dessa forma, dado um \termo{cseg}
$P\:\langle5\:9\:6\:8\rangle$, são considerados equivalentes os
\termo{cseg} $\langle1\:5\:2\:3\rangle$, $\langle0\:10\:4\:7\rangle$,
$\langle0\:3\:1\:2\rangle$, entre outros. Todos eles têm
$\langle0\:3\:1\:2\rangle$ como forma normal.

A operação de inversão de um contorno $P$ de ordem $q$ é representada
por $IP$, e é matematicamente calculada como mostra a equação
\ref{eq:operacao-de-inversao}.

\begin{equation}
  \label{eq:operacao-de-inversao}
  IP_n=(q-1-P_n)
\end{equation}

Portanto, dado um \termo{cseg} $P\:\langle5\:9\:6\:8\rangle$ (de ordem
10), $IP_0=(10-1-P_0)$. Logo, $IP_0=4$. Aplicando-se a mesma idéia aos
outros elementos chegamos ao contorno
$IP\:\langle4\:0\:3\:1\rangle$. É possível ainda relacionar a inversão
entre elementos através da matriz de comparação, como mostra a equação
$COM(P_a,P_b)=-COM(IP_a,IP_b)$.

A operação de translação de um \termo{csubseg} de $n$ \termo{c-pitch}
distintos não numerados de $0$ a $(n-1)$ consiste na renumeração com
$0$ para o \termo{c-pitch} mais grave e $(n-1)$ para o mais
agudo. Operações de retrogradação e inversão são idênticas às de
alturas de notas.

A forma prima de um \termo{cseg} é calculada fazendo-se três operações
\cite{marvin.ea87:relating}. Sendo $a$ o primeiro \termo{c-pitch}, $b$
o último, e $n$ o número total de \termo{c-pitch}, realiza-se:
\begin{enumerate}
\item translação, caso o \termo{cseg} não esteja em sua forma normal;
%% FIXME: tirar o "; e"
\item inversão, caso $a$ seja maior que $[(n-1) - b]$ (vide equação
  \ref{eq:operacao-de-inversao});
\item retrogradação, caso $a$ seja maior que $b$.
\end{enumerate}

A função de similaridade de contornos \termo{CSIM} mede o grau de
similaridade entre dois \termo{cseg} de mesma cardinalidade. Ela
compara as posições do triângulo superior direito da
\termo{COM-matrix} de ambos os \termo{cseg}. O valor de \termo{CSIM} é
representado pelo quociente entre a soma de posições equivalentes e o
número total de posições da \termo{COM-matrix}. Os valores da
\termo{CSIM} variam entre 0 e 1, sendo 1 o grau máximo de
similaridade. Considerando, por exemplo os \termo{cseg} da tabela
\ref{tab:matriz-exemplos} como $P\:\langle5\:9\:6\:8\rangle$,
$Q\:\langle5\:7\:6\:8\rangle$ e $R\:\langle3\:0\:5\:1\rangle$, $P$ tem
uma similaridade muito maior com Q, do que com R, já que
$CSIM(P,Q)=0,83$, e $CSIM(P,R)=0,16$.

Os Intervalos de Contorno (\termo{CI}) representam as relações entre
\termo{c-pitch} de uma \termo{CC} e podem ser entendidos de duas
formas: guardando o valor entre os \termo{c-pitch}
\cite{friedmann85:methodology}, ou guardando apenas a direção, mas não
o valor da diferença entre os \termo{c-pitch}
\cite{morris93:directions}. Por exemplo, para um mesmo \termo{cseg}
$P\:\langle0\:3\:1\:2\rangle$, o \termo{CI} entre $P_1$ e $P_2$ para
Friedmann é -2, e para Morris, ``$-$''.

A Série de Intervalos de Contornos (\termo{CIS}) considera, além da
direção dos movimentos, o \termo{CI} entre todos os \termo{c-pitch}
adjacentes de um \termo{cseg}. Por exemplo, dado um \termo{cseg}
$P\:\langle5\:9\:6\:8\rangle$, sua \termo{CIS} será
$\langle+4,-3,+2\rangle$. O Vetor de Intervalos de Contornos
(\termo{CIA}) descreve a freqüência de cada tipo de \termo{CI} em uma
\termo{CC}. Por exemplo, o \termo{cseg} $P\:\langle0\:3\:1\:2\rangle$
tem \termo{CIA} $\langle2,1,1/1,1,0\rangle$. Os dígitos da esquerda da
barra representam os \termo{CI} ascendentes em ordem crescente, e os
dígitos da direita os \termo{CI} descendentes em ordem crescente de
valor absoluto. \cite{friedmann85:methodology}.

\section{O software}
\label{sec:o-software}

\section{Programando para aprender}
\label{sec:progr-para-aprend}

\section{Composição com contornos}
\label{sec:comp-com-cont}

\note{parágrafo solto}
Na música pré-serial de Webern, por exemplo, na falta de outros
sistemas dominantes de organização de alturas, contorno representa um
fator estrutural igual em importância às relações da Teoria dos
Conjuntos \cite[p. 157]{clifford95:contour}.

\note{parágrafo solto}
Uma composição pode ser entendida como uma matriz de características
rítmicas, de classe de notas, de contornos e de timbre. As relações
entre contorno e classe de notas podem ser abordadas de duas maneiras:
considerando possíveis associações entre conjuntos e contornos, ou
tratando conjuntos de notas e contornos de forma autônoma

\renewcommand{\refname}{Referências Bibliográficas}
\bibliography{music-harmony-and-theory,music-history,melodic-contour,music-analysis,music-scores,music-perception,composition,programs}
\bibliographystyle{kchicago}

\end{document}
