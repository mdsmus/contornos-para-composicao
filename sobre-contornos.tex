\documentclass{article}
\usepackage{anppom2008}

% usar para termos estrangeiros
\newcommand{\eng}[1]{\textit{#1}}

% usar para nomes de obras
\newcommand{\opus}[1]{\textit{#1}}

% usar para nomes de termos
\newcommand{\termo}[1]{\textit{#1}}

\newcommand{\ok}{
  \multicolumn{1}{>{\columncolor[gray]{.6}}c}{}
}

\newcommand{\tri}[1]{
  #1\textsuperscript{o} t
}


\begin{document}
\graphicspath{{figs-out/}{out/}}

\title{Software para processar contornos}
\author{Marcos da Silva Sampaio e Pedro Kröger}{Universidade Federal
  da Bahia}{{mdsmus,pedro.kroeger}@gmail.com}{genos.mus.br}

\begin{sumario}
  Resumo do texto com até 100 palavras.  
\end{sumario}

\keywords{Contornos, Computação musical, Composição, Lisp}

\section{Introdução}
\label{sec:introducao}

%% falar sobre criação de material a partir de operações e falar de
%% operações de contorno
Ao longo da história da música compositores têm criado material
composicional a partir de operações como transposição, inversão e
retrogradação de estruturas como motivos, conjuntos ou séries de
notas.

O presente trabalho faz parte de uma pesquisa maior na qual é proposta
a criação de material composicional a partir de operações com
contornos melódicos.

Um software para processar contornos melódicos e retornar operações
pode contribuir significativamente com a pesquisa de composição a
partir de contornos. O Processador de Classe de Notas
\cite{oliveira01:pcn} é um exemplo de software criado para o mesmo
fim.

Contorno pode ser definido como o perfil, desenho ou formato de um
objeto. Pode ser bidimensional e associar altura a comprimento,
largura ou tempo. Em música contornos podem ser associados a altura,
densidade, ritmo, complexidade rítmica, homogeneidade orquestral,
amplitude de harmônicos, intensidade, etc. Contornos melódicos estão
relacionados com movimento de altura em função do tempo.

Diversas definições de contorno melódico foram feitas por autores como
Piston \cite{piston59:harmony}, Toch \cite{toch77:shaping}, Schönberg
\cite{schonberg:fundamentals}, Adams \cite{adams76:melodic}, Marvin e
Laprade \cite{marvin.ea87:relating}, Morris
\cite{morris87:composition}, Clifford \cite{clifford95:contour} e
Beard \cite{beard03:contour}. Cada uma destas definições está
relacionada com o objetivo do trabalhos destes autores. No presente
estudo preferimos entender que contorno melódico é o conjunto ordenado
dos movimentos ascendentes, descendentes, e nulos entre as notas de
uma melodia.

Em composição o contorno melódico é um elemento de nível médio de
abstração, entre altura, duração e timbre, e motivos e frases. Sua
presença na música é tão comum quanto a de qualquer outro elemento
citado.

A idéia de preservação de contorno e variação de intervalos entre
notas é encontrada em diferentes situações musicais. Há adequação de
notas à tonalidade em respostas tonais de fugas, em mudanças de modo
em peças do tipo tema e variações, em \eng{leitmotif} e idéias fixas,
citando apenas alguns exemplos
\cite[p. 29]{morris87:composition}. Outras obras têm motivos cujos
intervalos são pouco a pouco expandidos ou contraídos, como por
exemplo ocorre no início da \opus{Música para Cordas, Percussão e
  Celesta}, de Béla Bartók.

No campo da percepção musical contorno melódico é uma importante
característica para o reconhecimento de melodias familiares
\cite[p. 136]{dowling.ea86:music}. Alguns teóricos reconhecem que
ouvintes têm maior acuidade na percepção de semelhança de contornos do
que na semelhança de alturas. Por isso novas teorias para comparação
de contornos se tornaram necessárias à área da Análise Musical
\cite[p. 226]{marvin.ea87:relating}.

%% falar do uso de contorno como de determinante composicionais

Na falta de outros sistemas dominantes de organização de alturas,
contorno representa um fator estrutural igual em importância às
relações da Teoria dos Conjuntos \cite[p. 157]{clifford95:contour}.
Dessa forma a análise com uso das ferramentas das teorias de contornos
se mostra adequada.

%% falar da proposta do mestrado: usar operações de contorno melódico
%% para gerar material composicional


\section{Teorias de contornos}
\label{sec:teorias-de-contornos}

Teorias que visam sistematizar o estudo de contornos têm sido
desenvolvidas desde a década de 1980 por Friedmann
\cite{friedmann85:methodology,friedmann87:response}, Morris
\cite{morris87:composition,morris93:directions}, Marvin e Laprade
\cite{marvin.ea87:relating}, Clifford \cite{clifford95:contour},
Polansky e Bassein \cite{polansky.ea92:possible}, Quinn
\cite{quinn97:fuzzy} e Beard \cite{beard03:contour}.

A música pode ser entendida a partir de diferentes espaços: de altura,
de contorno, de timbre, de ritmo \cite{morris87:composition}. De todos
esses, o espaço de contorno (\eng{c-space}) é o mais importante para o
nosso estudo. Trata-se de uma abstração de espaço musical que consiste
em elementos organizados do grave para o agudo desconsiderando os
intervalos exatos entre eles.

Uma composição pode ser entendida como uma matriz de características
rítmicas, de classe de notas, de contornos e de timbre. As relações
entre contorno e classe de notas podem ser abordadas de duas maneiras:
considerando possíveis associações entre conjuntos e contornos, ou
tratando conjuntos de notas e contornos de forma autônoma
\cite{friedmann85:methodology}.

Contornos são representados por letras maiúsculas e seus elementos por
numerais subscritos. Estes numerais representam a posição destes
elementos em ordem temporal de aparecimento no contorno. Por exemplo,
em um contorno $P\:\langle5\:9\:6\:8\rangle$, $P_0 = 5$, $P_1 = 9$,
$P_2 = 6$, e $P_3 = 8$.

\section{O software}
\label{sec:o-software}

\section{Programando para aprender}
\label{sec:progr-para-aprend}

\section{Composição com contornos}
\label{sec:comp-com-cont}

\renewcommand{\refname}{Referências Bibliográficas}
\bibliography{music-harmony-and-theory,music-history,melodic-contour,music-analysis,music-scores,music-perception,composition,programs}
\bibliographystyle{kchicago}

\end{document}
