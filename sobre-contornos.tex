\documentclass[brazil]{article}
\usepackage{anppom2008}

% usar para termos estrangeiros
\newcommand{\eng}[1]{\textit{#1}}

% usar para nomes de obras
\newcommand{\opus}[1]{\textit{#1}}

% usar para nomes de termos
\newcommand{\termo}[1]{\textit{#1}}

\newcommand{\goiaba}{\opus{Goiaba}}

\begin{document}
\graphicspath{{figs-out/}{out/}}

\title{Software para processar contornos}
\author{}{}{}{}
%\author{Marcos da Silva Sampaio e Pedro Kröger}{Universidade Federal
%  da Bahia}{{mdsmus,pedro.kroeger}@gmail.com}{genos.mus.br}

\begin{sumario}
  Resumo do texto com até 100 palavras.  
\end{sumario}

\keywords{Contornos, Computação musical, Composição, Lisp}

%% divisão de trabalho

%% por que contornos são importantes
%%%% na análise
%%%% na composição

%% criação de um programa para ajudar na compreensão de contornos

%% explicar as teorias de contornos

%% dar uma visão geral do programa
%%%% mostrar combinações

%% falar sobre criação de material a partir de operações e falar de
%% operações de contorno

\section{Introdução}
\label{sec:introducao}

Contorno pode ser definido como o perfil, desenho ou formato de um
objeto. Pode ser bidimensional e associar altura a comprimento,
largura ou tempo. Em música contornos podem ser associados a altura,
densidade, ritmo, complexidade rítmica, homogeneidade orquestral,
amplitude de harmônicos, intensidade, etc. Contornos melódicos estão
relacionados com movimento de altura em função do tempo.

Há diversas definições de contorno melódico na literatura
\cite{piston59:harmony,toch77:shaping,schonberg:fundamentals,adams76:melodic,marvin.ea87:relating,morris87:composition,clifford95:contour,beard03:contour}. Cada
uma destas definições está relacionada com o objetivo do trabalhos de
seus autores. No presente estudo preferimos entender que contorno
melódico é um conjunto ordenado dos movimentos ascendentes,
descendentes, e nulos entre as notas de uma melodia.

%% por que contornos são importantes

O estudo de contornos é importante porque, assim como conjuntos de
notas e motivos, contornos ajudam a dar coerência musical a uma
obra. Eles representam estruturas musicais manipuláveis através de
várias operações como inversão e retrogradação. Contornos podem ser
abordados tanto do ponto de vista da análise, quanto da composição.

Na música pré-serial de Webern, por exemplo, na falta de outros
sistemas dominantes de organização de alturas, contorno representa um
fator estrutural igual em importância às relações da Teoria dos
Conjuntos \cite[p. 157]{clifford95:contour}.

A idéia de preservação de contorno e variação de intervalos entre
notas é encontrada em diferentes situações musicais. Há adequação de
notas à tonalidade em respostas tonais de fugas, em mudanças de modo
em peças do tipo tema e variações, em \eng{leitmotif} e idéias fixas,
citando apenas alguns exemplos
\cite[p. 29]{morris87:composition}. Outras obras têm motivos cujos
intervalos são pouco a pouco expandidos ou contraídos, como por
exemplo ocorre no início da \opus{Música para Cordas, Percussão e
  Celesta}, de Béla Bartók.

No campo da percepção musical contorno melódico é uma importante
característica para o reconhecimento de melodias familiares
\cite[p. 136]{dowling.ea86:music}. Alguns teóricos reconhecem que
ouvintes têm maior acuidade na percepção de semelhança de contornos do
que na semelhança de alturas. Por isso novas teorias para comparação
de contornos se tornaram necessárias à área da Análise Musical
\cite[p. 226]{marvin.ea87:relating}.

%% criação de um programa para ajudar na compreensão de contornos

O presente trabalho faz parte de uma pesquisa maior na qual é proposta
a criação de material composicional a partir de operações com
contornos melódicos. Um software para processar contornos melódicos e
retornar operações pode contribuir significativamente com a citada
pesquisa. O Processador de Classe de Notas \cite{oliveira01:pcn} é um
exemplo de software criado para o mesmo fim.

\section{Teorias de contornos}
\label{sec:teorias-de-contornos}

Teorias que visam sistematizar o estudo de contornos dispõem de várias
operações para mapeamento e comparação de contornos
\cite{friedmann85:methodology,friedmann87:response,morris87:composition,morris93:directions,marvin.ea87:relating,clifford95:contour,polansky.ea92:possible,quinn97:fuzzy,beard03:contour}.

A música pode ser entendida a partir de diferentes espaços, como de
altura e de contorno \cite{morris87:composition}. O espaço de contorno
(\eng{c-space}) é uma abstração de espaço musical que consiste em
elementos organizados do grave para o agudo desconsiderando os
intervalos exatos entre eles.

Uma composição pode ser entendida como uma matriz de características
rítmicas, de classe de notas, de contornos e de timbre. As relações
entre contorno e classe de notas podem ser abordadas de duas maneiras:
considerando possíveis associações entre conjuntos e contornos, ou
tratando conjuntos de notas e contornos de forma autônoma
\cite{friedmann85:methodology}.

Por definição contornos são ordenados, e representados por letras
maiúsculas e seus elementos por numerais subscritos. Estes numerais
representam a posição destes elementos em ordem temporal de
aparecimento no contorno \cite{marvin.ea87:relating}. Por exemplo, em
um contorno $P\:\langle5\:9\:6\:8\rangle$, $P_0$ é igual a 5, $P_1$ é
igual a 9, $P_2$ é igual a 6, e $P_3$ é igual a 8.

\eng{C-space} pode ser entendido como um grande conjunto de
\eng{c-pitch} (alturas de contorno). Cada contorno contido em um
\eng{c-space} é chamado de \eng{cseg} (segmento de
contorno)\footnote{Trata-se de uma idéia semelhante à da geometria, de
  reta e segmento de reta, onde o \eng{c-space} seria análogo à reta,
  e o \eng{cseg} ao segmento de reta.}. Subconjuntos de \eng{cseg} são
chamados \eng{csubseg}, e podem conter elementos contíguos ou não do
\eng{cseg} que os contém.

Relações de similaridade \cite{marvin.ea87:relating} são analisadas
com a função de comparação \eng{com}, com a matriz de comparação
(\eng{com-matrix}), com as formas normal e prima, com a classe de
contorno (\eng{cc}), com a função de similaridade \eng{csim}, e com a
função de contorno embutido \eng{cemb}. As operações com contorno são
identidade (P), inversão (I), retrogradação (R), retrogradação da
inversão (RI) e translação.

%% FIXME: tirar parênteses da referência à equação
A função de comparação \eng{com} mede a diferença de registro entre
dois elementos, ou seja, informa se um elemento é mais agudo, mais
grave ou de mesma altura que outro. O valor de \eng{com} é o sinal
``$+$'' se $a$ é menor que $b$; ``$-$'' se $a$ é maior que $b$; e
``$0$'' se $a$ é igual a $b$. Por exemplo, no contorno
$P\:\langle5\:9\:6\:8\rangle$, o valor de $COM(P_0,P_1)$ é o sinal
``$+$'', o de $COM(P_1,P_2)$ é ``$-$'', e o valor de $COM(P_3,P_0)$ é
o sinal ``$+$''. Esta medida de comparação pode ser invertida de modo
que a comparação entre dois elementos é igual ao inverso da comparação
destes elementos em ordem reversa. Esta idéia pode ser melhor
entendida observando-se a equação (\ref{eq:inversao-da-comparacao}).

\begin{equation}
  \label{eq:inversao-da-comparacao}
  COM(a,b)=-COM(b,a)
\end{equation}

A \termo{com-matrix} é uma matriz bidimensional que compara um
\termo{cseg} com ele próprio. Ela mostra os resultados da função de
comparação \termo{com} para todos os \termo{c-pitch} de um
\termo{cseg}. Cada posição da matriz é representada de modo genérico
por $E_(P_x,P_y)$, onde $E$ representa a própria matriz, $P$
representa o \termo{cseg} que dá origem à matriz, $P_x$ representa
genericamente um \termo{c-pitch} do \termo{cseg} $P$ localizado
horizontalmente na matriz, e $P_y$ representa um \termo{c-pitch} de
$P$ localizado verticalmente na matriz. A tabela \ref{tab:matriz-5968}
contém a matriz de comparação de um \termo{cseg}
$P\:\langle5\:9\:6\:8\rangle$, e outros exemplos de \termo{com-matrix}
podem ser vistos na tabela \ref{tab:matriz-exemplos}.

\begin{table}
  \centering
  \subtable[cseg $P\langle5\:9\:6\:8\rangle$]{
    \begin{tabular}{c|cccc}
      & $5$ & $9$ & $6$ & $8$ \\
      \hline
      $5$ & $0$ & $+$ & $+$ & $+$ \\
      $9$ & $-$ & $0$ & $-$ & $-$ \\
      $6$ & $-$ & $+$ & $0$ & $+$ \\
      $8$ & $-$ & $+$ & $-$ & $0$
    \end{tabular}
    \label{tab:matriz-5968}
  }
  \subtable[cseg $Q\langle5\:7\:6\:8\rangle$]{
    \begin{tabular}{c|cccc}
      & $5$ & $7$ & $6$ & $8$ \\
      \hline
      $5$ & $0$ & $+$ & $+$ & $+$ \\
      $7$ & $-$ & $0$ & $-$ & $+$ \\
      $6$ & $-$ & $+$ & $0$ & $+$ \\
      $8$ & $-$ & $-$ & $-$ & $0$
    \end{tabular}
    \label{tab:matriz-5968}
  }
  \subtable[cseg $R\langle3\:0\:5\:1\rangle$]{
    \begin{tabular}{c|cccc}
      & $3$ & $0$ & $5$ & $1$ \\
      \hline
      $3$ & $0$ & $-$ & $+$ & $-$ \\
      $0$ & $+$ & $0$ & $+$ & $+$ \\
      $5$ & $-$ & $-$ & $0$ & $-$ \\
      $1$ & $+$ & $-$ & $+$ & $0$
    \end{tabular}
    \label{tab:matriz-3051}
  }
  \label{tab:matriz-exemplos}
  \caption{Exemplos de \eng{COM-matrix}}
\end{table}

Esta matriz tem uma diagonal principal zero. Esta diagonal estabelece
uma simetria entre elementos da matriz. As diagonais superiores são
chamadas de $INT_n$, onde $n$ é o número da diagonal: 1 para a
superior mais próxima da diagonal zero, 2 para a seguinte, 3 para a
posterior e assim por diante.

A diagonal $INT_1$ traz comparações de registros entre elementos
adjacentes do \termo{cseg}. Em $P\:\langle5\:9\:6\:8\rangle$, por
exemplo, $INT_1=\langle+\:-\:+\rangle$, ou seja, o movimento melódico
é ascendente entre 5 e 9, descendente entre 9 e 6, e ascendente entre
6 e 8. Esta comparação é feita também com uma operação conhecida como
\termo{cas} \footnote{Em teorias de contornos não há consenso em
  relação a terminologia. Isto se dá por terem sido desenvolvidas por
  autores diferentes. Por isso há idéias semelhantes com nomes
  diferentes, como $INT_1$ e \termo{cas}
  \cite{friedmann87:response}.}.

A classe de contorno (\termo{cc}) é uma operação importante para a
verificação de similaridade entre contornos. Assim como a forma
normal, a \termo{cc} é obtida numerando-se ordenadamente todos os
\termo{c-pitch} de $0$ a $(n-1)$, sendo $n$ o número de
\termo{c-pitch} do \termo{cseg}. Uma \termo{cc} engloba todos os
contornos considerados equivalentes.

Dois ou mais contornos são considerados equivalentes se geram uma
mesma \termo{com-matrix}, ou seja, se mantêm a mesma estrutura de
registro entre suas notas. Dessa forma, dado um \termo{cseg}
$P\:\langle5\:9\:6\:8\rangle$, são considerados equivalentes os
\termo{cseg} $\langle1\:5\:2\:3\rangle$, $\langle0\:10\:4\:7\rangle$,
$\langle0\:3\:1\:2\rangle$, entre outros. Todos eles têm
$\langle0\:3\:1\:2\rangle$ como forma normal.

A operação de inversão de um contorno $P$ de ordem $q$ é representada
por $IP$, e é matematicamente calculada como mostra a equação
(\ref{eq:operacao-de-inversao}).

\begin{equation}
  \label{eq:operacao-de-inversao}
  IP_n=(q-1-P_n)
\end{equation}

Portanto, dado um \termo{cseg} $P\:\langle5\:9\:6\:8\rangle$ (de ordem
10), $IP_0=(10-1-P_0)$. Logo, $IP_0=4$. Aplicando-se a mesma idéia aos
outros elementos chegamos ao contorno
$IP\:\langle4\:0\:3\:1\rangle$. É possível ainda relacionar a inversão
entre elementos através da matriz de comparação, como mostra a equação
(\ref{eq:operacao-de-inversao-na-comparacao}).

\begin{equation}
  \label{eq:operacao-de-inversao-na-comparacao}
  COM(P_a,P_b)=-COM(IP_a,IP_b)
\end{equation}

A operação de translação de um \termo{csubseg} de $n$ \termo{c-pitch}
distintos não numerados de $0$ a $(n-1)$ consiste na renumeração com
$0$ para o \termo{c-pitch} mais grave e $(n-1)$ para o mais
agudo. Operações de retrogradação e inversão são idênticas às de
alturas de notas.

A forma prima de um \termo{cseg} é calculada fazendo-se três operações
\cite{marvin.ea87:relating}. Sendo $a$ o primeiro \termo{c-pitch}, $b$
o último, e $n$ o número total de \termo{c-pitch}, realiza-se:
\begin{enumerate}
\item translação, caso o \termo{cseg} não esteja em sua forma normal;
%% FIXME: tirar o "; e"
\item inversão, caso $a$ seja maior que $[(n-1) - b]$ (vide equação
  (\ref{eq:operacao-de-inversao})); e
\item retrogradação, caso $a$ seja maior que $b$.
\end{enumerate}

A função de similaridade de contornos \termo{csim} mede o grau de
similaridade entre dois \termo{cseg} de mesma cardinalidade. Ela
compara as posições do triângulo superior direito da
\termo{com-matrix} de ambos os \termo{cseg}. O valor de \termo{csim} é
representado pelo quociente entre a soma de posições equivalentes e o
número total de posições da \termo{com-matrix}. Os valores da
\termo{csim} variam entre 0 e 1, sendo 1 o grau máximo de
similaridade. Considerando, por exemplo os \termo{cseg} da tabela
\ref{tab:matriz-exemplos} como $P\:\langle5\:9\:6\:8\rangle$,
$Q\:\langle5\:7\:6\:8\rangle$ e $R\:\langle3\:0\:5\:1\rangle$, $P$ tem
uma similaridade muito maior com Q, do que com R, já que
$CSIM(P,Q)=0,83$, e $CSIM(P,R)=0,16$.

Os Intervalos de Contorno (\termo{ci}) representam as relações entre
\termo{c-pitch} de uma \termo{cc} e podem ser entendidos de duas
formas. Para Friedmann \cite{friedmann85:methodology} os \termo{ci}
guardam o valor entre os \termo{c-pitch}. Para Morris
\cite{morris93:directions} os \termo{ci} guardam apenas a direção, mas
não o valor da diferença entre os \termo{c-pitch}. Por exemplo, para
um mesmo \termo{cseg} $P\:\langle0\:3\:1\:2\rangle$, o \termo{ci}
entre $P_1$ e $P_2$ para Friedmann é -2, e para Morris, ``$-$''.

A Série de Intervalos de Contornos (\termo{cis}) considera, além da
direção dos movimentos, o \termo{ci} entre todos os \termo{c-pitch}
adjacentes de um \termo{cseg}. Por exemplo, dado um \termo{cseg}
$P\:\langle5\:9\:6\:8\rangle$, sua \termo{cis} será
$\langle+4,-3,+2\rangle$.

O Vetor de Intervalos de Contornos (\termo{cia}) descreve a freqüência
de cada tipo de \termo{ci} em uma \termo{cc}. Por exemplo, o
\termo{cseg} $P\:\langle0\:3\:1\:2\rangle$ tem \termo{cia}
$\langle2,1,1/1,1,0\rangle$. Os dígitos da esquerda da barra
representam os \termo{ci} ascendentes em ordem crescente, e os dígitos
da direita os \termo{ci} descendentes em ordem crescente de valor
absoluto.

\section{O software}
\label{sec:o-software}

\section{Programando para aprender}
\label{sec:progr-para-aprend}

\section{Composição com contornos}
\label{sec:comp-com-cont}

\renewcommand{\refname}{Referências Bibliográficas}
\bibliography{music-harmony-and-theory,music-history,melodic-contour,music-analysis,music-scores,music-perception,composition,programs}
\bibliographystyle{kchicago}

\end{document}
