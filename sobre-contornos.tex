\documentclass{article}
\usepackage{anppom2008}

% usar para termos estrangeiros
\newcommand{\eng}[1]{\textit{#1}}

% usar para nomes de obras
\newcommand{\opus}[1]{\textit{#1}}

% usar para nomes de termos
\newcommand{\termo}[1]{\textit{#1}}

\newcommand{\ok}{
  \multicolumn{1}{>{\columncolor[gray]{.6}}c}{}
}

\newcommand{\tri}[1]{
  #1\textsuperscript{o} t
}


\begin{document}
\graphicspath{{figs-out/}{out/}}

\title{Software para processar contornos}
\author{Marcos da Silva Sampaio e Pedro Kröger}{Universidade Federal
  da Bahia}{{mdsmus,pedro.kroeger}@gmail.com}{genos.mus.br}

\begin{sumario}
  Resumo do texto com até 100 palavras.  
\end{sumario}

\keywords{Contornos, Computação musical, Composição, Lisp}

\section{Introdução}
\label{sec:introducao}

Teorias que visam sistematizar o estudo de contornos têm sido
desenvolvidas desde a década de 1980 por Michael Friedmann
\cite{friedmann85:methodology,friedmann87:response}, Robert Morris
\cite{morris87:composition,morris93:directions}, Elizabeth Marvin e
Paul Laprade \cite{marvin.ea87:relating}, e Robert Clifford
\cite{clifford95:contour}.

Diversas definições de contorno melódico foram feitas por autores como
Piston, Toch, Schönberg, Adams, Marvin, Morris, Clifford e Beard. Cada
uma destas definições está relacionada com o objetivo do trabalhos
destes autores. Para o presente estudo preferimos entender que
contorno melódico é o conjunto ordenado dos movimentos ascendentes,
descendentes, e nulos entre as notas de uma melodia.

A idéia de preservação de contorno e variação de intervalos entre
notas é encontrada em diferentes situações musicais. Há adequação de
notas à tonalidade em respostas tonais de fugas, em mudanças de modo
em peças do tipo tema e variações, em \eng{leitmotif} e idéias fixas,
citando apenas alguns exemplos
\cite[p. 29]{morris87:composition}. Outras obras têm motivos cujos
intervalos são pouco a pouco expandidos ou contraídos, como por
exemplo ocorre no início da \opus{Música para Cordas, Percussão e
  Celesta}, de Béla Bartók.

No campo da percepção musical contorno melódico é uma importante
característica para o reconhecimento de melodias familiares
\cite[p. 136]{dowling.ea86:music}. Alguns teóricos reconhecem que
ouvintes têm maior acuidade na percepção de semelhança de contornos do
que na semelhança de alturas. Por isso novas teorias para comparação
de contornos se tornaram necessárias à área da Análise Musical
\cite[p. 226]{marvin.ea87:relating}.

\section{O software}
\label{sec:o-software}

\section{Programando para aprender}
\label{sec:progr-para-aprend}

\section{Composição com contornos}
\label{sec:comp-com-cont}

\renewcommand{\refname}{Referências Bibliográficas}
\bibliography{music-harmony-and-theory,music-history,melodic-contour,music-analysis,music-scores,music-perception}
\bibliographystyle{kchicago}

\end{document}
